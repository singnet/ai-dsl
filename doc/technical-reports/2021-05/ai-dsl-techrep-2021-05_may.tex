%% Technical Report for the work on the AI-DSL over the period of
%% March to May 2021.

\documentclass[]{report}
\usepackage{url}

\usepackage[textsize=footnotesize]{todonotes}
\newcommand{\kabir}[2][]{\todo[color=yellow,author=kabir, #1]{#2}}
\usepackage{hyperref}

\begin{document}

\title{AI-DSL Technical Report (February to May 2021)}
\author{Nil Geisweiller, Kabir Veitas, Eman Shemsu Asfaw, Samuel Roberti}
\maketitle

\begin{abstract}
  Based on \cite{GoertzelGeisweillerBlog}.
\end{abstract}

\tableofcontents

\chapter{Nil's work}

\chapter{AI-DSL Ontology (Kabir's work)}

\section{Design requirements}

Based on:
\begin{enumerate}
	\item the summary of \href{https://github.com/nunet-io/ai-dsl-ontology/wiki}{initial discussions about requirements}
	\item possibly augmented by later research.
\end{enumerate}

\section{Domain model considerations}

Explanation of NuNet fake-news-detector domain model and related considerations  making the first ontology, based on:
\begin{itemize}
	\item presentation on NuNet service discovery;
	\item augmented by developments on the system over last month;
	\item using ontology for agent communication in decentralized computing systems, based on \cite{YvesHellenschmidt2002};
\end{itemize}

\kabir[inline]{The domain model may need to be presented somewhere else, as it may be related to other sections besides AI-DSL ontology}

\section{Choice of existing ontologies}

Based on:
\begin{enumerate}
	\item discussion on \href{reusing existing ontologies}{https://github.com/singnet/ai-dsl/discussions/18} for the choice of SUMO and KIF;
	\item Usage of:
	\begin{itemize}
		\item Upper level SUMO ontology (\href{Merge.kif}{https://github.com/ontologyportal/sumo/blob/master/Merge.kif});
		\item Middle level SUMO ontology (\href{Mid-level-ontology.kif}{https://github.com/ontologyportal/sumo/blob/master/Mid-level-ontology.kif});
		\item Distributed computing hardware domain ontology in SUO-KIF (\href{QoSontology.kif}{https://github.com/ontologyportal/sumo/blob/master/QoSontology.kif});
		\item \href{Software ontology}{https://github.com/allysonlister/swo} in OWL. In the long term, it may be ideal to develop a converter for converting it to KIF, since OWL may be representable in KIF \cite{martin_translations_nodate} using \href{OWL API}{https://github.com/owlcs/owlapi}; For the purpose of the ontology prototype, we will manually select parts of the this ontology in order to build the prototype and write them in SUO-KIF format;
	\end{itemize}
\end{enumerate}

\section{Tools}

Intro to Sigma, SigmajEdit, etc. and how to install them.

\section{AI-DSL ontology prototoype}

The prototype will be the fake-news-detector leaf ontology based on the above listed upper and middle ontologies (SUMO) and domain ontologies of computer hardware and sofware.

\section{Combining ontology with Idris}

\kabir[inline]{It would be good to have a section explaining ideas about that, but I cannot do this alone, so probably the best is to reserve it for the end of the monht, when all the other aspects of AI-DSL project (including Idris) are explained.}

\section{Summary of results and future work}



\chapter{Eman's work}

\chapter{Sam's work}



\bibliographystyle{splncs04}
\bibliography{local}

\end{document}
